\documentclass[11pt]{article}

\usepackage{sectsty}
\usepackage{graphicx}
\usepackage{amsmath}
\usepackage{amssymb} 


% Margins
\topmargin=-0.45in
\evensidemargin=0in
\oddsidemargin=0in
\textwidth=6.5in
\textheight=9.0in
\headsep=0.25in

\graphicspath{ {images/} }

\begin{document}

\title{ Ingeniería del Conocimiento: Entrega de Clase 3}
\author{José Luis Molina Aguilar\newline }
\date{March 10, 2022}


\begin{center}
  \makebox[\textwidth]{\includegraphics[width=15cm]{download.png}}
\end{center}

\maketitle	

\begin{itemize}
  \item Ejercicio Practico de Adquisicion de Conocimiento
  \item[] Partiendo de la siguiente tabla donde podemos recomendar una rama a partir de la informacion de unos alumnos actuando nosotros como expertos, aplicamos un algoritmo de aprendizaje para obtener el 
  arbol de clasificación y de ahi obtener las reglas de decision asociadas al arbol obtenido

  \begin{table}[htbp]
    \centering
    \small
    \setlength\tabcolsep{0.75pt}
    \begin{center}
    \begin{tabular}{|c|c|c|c|c|c|c|p{15mm}|c| }
      \hline
    Caso  &  GustaMat  & Quiere trabajar  & Nota media & Hardware &Trabajador & GustaProgramar & Teóricas \newline prácticas  & Rama aconsejada\\ \hline  
    Alumno 1 & Si & Docencia & Alta & No & Mucho & Si & teóricas & CSI\\
    Alumno 2 & No & EmpresaPública & Media & No & Normal & No & prácticas & SI\\
    Alumno 3 & Si & EmpresaPrivada & Media & Si & Normal & Si & ambas & CSI\\
    Alumno 4  & No & EmpresaPrivada & Baja & No & Poco & Si & ambas & SI\\
    Alumno 5 & No & le da igual & Alta & Si & Mucho & Si & prácticas & IC\\
    Alumno 6 & Si & Docencia & Media & No & Poco & Si & teóricas & SI\\
    Alumno 7 & No & Docencia & Alta & No & Normal & Si & prácticas & IS\\
    Alumno 8 & No & EmpresaPública & Baja & Si & Normal & No & ambas & TI\\
    Alumno 9 & Si & EmpresaPrivada & Alta & No & Normal & Si & ambas & CSI\\
    Alumno 10 & No & EmpresaPública & Baja & No & Poco & Si & prácticas & TI\\
    Alumno 11 & Si & le da igual & Baja & No & Poco & Si & ambas & TI\\
    Alumno 12 & Si & EmpresaPrivada & Alta & Si & Normal & No & prácticas & IC\\
    Alumno 13 & No & EmpresaPrivada & Baja & Si & Poco & Si & prácticas & TI\\
    Alumno 14 & No & EmpresaPrivada & Alta & Si & Normal & Si & ambas & TI\\
    Alumno 15 & Si & Docencia & Media & No & Mucho & Si & teóricas & CSI\\
    Alumno 16 & No & Docencia & Media & No & Normal & No & teóricas & SI\\
    Alumno 17 & No & EmpresaPública & Media & Si & Poco & No & prácticas & IC\\
    Alumno 18 & Si & le da igual & Media & No & Normal & Si & ambas & SI\\
    Alumno 19 & No & Empresa Pública & Alta & Si & Mucho & Si & prácticas & IC\\
    Alumno 20 & Si & le da igual & baja & No & Poco & No & teóricas & SI\\ \hline
    

    \end{tabular}
    \end{center}
    \end{table}

     \item[] INSERTAR IMAGEN DEL ARBOL
  \end{itemize}


\end{document}