\documentclass[12pt]{article}
\usepackage{amsmath}
\usepackage{array}
\usepackage[utf8]{inputenc}
\usepackage{graphicx}
\usepackage{amssymb}
\usepackage{listings}
\usepackage{hyperref}

\usepackage{lmodern}

% (2) specify encoding
\usepackage[T1]{fontenc}

% (3) load symbol definitions
\usepackage{textcomp}

\graphicspath{ {images/} }

\usepackage{color}

\definecolor{dkgreen}{rgb}{0,0.6,0}
\definecolor{gray}{rgb}{0.5,0.5,0.5}
\definecolor{mauve}{rgb}{0.58,0,0.82}

\lstset{frame=tb,
  language=Python,
  aboveskip=3mm,
  belowskip=3mm,
  showstringspaces=false,
  columns=flexible,
  basicstyle={\small\ttfamily},
  numbers=none,
  numberstyle=\tiny\color{gray},
  keywordstyle=\color{blue},
  commentstyle=\color{dkgreen},
  stringstyle=\color{mauve},
  breaklines=true,
  breakatwhitespace=true,
  tabsize=3
}
\begin{document}

\title{Practica 0 Aprendizaje Automatico}
\author{José Luis Molina Aguilar\newline }
\date{}


\maketitle

\begin{enumerate}
  \item[] Ejercicio 1: Base de datos IRIS.
  \begin{enumerate}
    \item[] Para poder obtener los datatos de la base de datos de Iris necesito exportarlos, para ello:
    \begin{lstlisting}
      from sklearn.datasets import load_iris. 
    \end{lstlisting}
    \item[] Luego necesitamos obtener las caracterisitcas de los datos, para ello podemos llamar a las miembres data y tarjet.
    \item[] Data es 
    \item[] Tarjet contiene
    \item[] Para obtener las caracterisitcas de los datos, concretamente para las columnas 1 y 3, lo que represetnaria el sepal width y petal width utilizaremos X[]
    \item[] Despues tendriamos que separar los diferentes tipos, en este caso por setosa, versicolor y virginica; sabemos que hay 50 de cada tipo lo que da un total de 150 datos en total
    \item[] Finalmente para realizar el grafico, pintaremos cada tipo anteriormente mencionado con un color diferente utilizando la funcion de matplotlib scatter, por ejemlo para pintar las setosa, utilizaremos  
  \end{enumerate} 
  \item[] Ejercicio 2: Entrenamiento
  \begin{enumerate}
    \item 
  \end{enumerate} 
  \item[] Ejercicio 3: Graficos en 2D, plots
  \begin{enumerate}
    \item[] Este ejercicio trata de representar funciones en graficos 2D, especificamente dibujaremos entre 0 y 4$\pi$, para poder dividir las coordenadas del eje X en 100 partes iguales utilizamos la funcion de linspace y despues simplemente para las 
    coordenadas del eje Y utilizaremos las funciones matematicas que nos piden.
    \item[] Para realizar los graficos una vez tenemos las coordenadas utilizaremos la funcion plot de matplotlib, cuyos parametros son simplemente tanto las coordenadas X e Y y algunos paramtros opcionales que nosotros en este caso utilizaremos como el color y el estilo de la linspace
    \item[] Finalmente delimitare el rango de valores que puede tener ambos ejex, para ello mediante plt.xlim o plt.ylim y pasandole los parametros entre [] separados por coma
  \end{enumerate} 
  \item[] Ejercicio 4: Graficos en 3D
  \begin{enumerate}
    \item[] Para este ejercicio primero que nada definiremos las dos funcions que vamos a dibujor, podemos hacerlo tanto con lambda como con funciones, en mi caso será la segunda opcion.
    \item[] Para especificar el grafico es en 3 dimensiones utilizaremos la funcion \verb|add_subplot| con el parametro de projection='3d'
    \item[] Para realizar las figuras utilizaremos la funcion \verb|plot_surface| con las coordenadas X, Y, Z, ademas se le pueden pasar mas argumentos para poder definir mas cosas como el mapa de color, las esquinas 
  \end{enumerate} 
\end{enumerate}

\end{document}
