%%%%%%%%%%%%%%%%%%%%%%%%%%%%%%%%%%%%%%%%%
% Short Sectioned Assignment LaTeX Template Version 1.0 (5/5/12)
% This template has been downloaded from: http://www.LaTeXTemplates.com
% Original author:  Frits Wenneker (http://www.howtotex.com)
% License: CC BY-NC-SA 3.0 (http://creativecommons.org/licenses/by-nc-sa/3.0/)
%%%%%%%%%%%%%%%%%%%%%%%%%%%%%%%%%%%%%%%%%

%----------------------------------------------------------------------------------------
%	PACKAGES AND OTHER DOCUMENT CONFIGURATIONS
%----------------------------------------------------------------------------------------

\documentclass[paper=a4, fontsize=11pt]{scrartcl} % A4 paper and 11pt font size

% ---- Entrada y salida de texto -----

%\usepackage[T1]{fontenc} % Use 8-bit encoding that has 256 glyphs
\usepackage[utf8]{inputenc}

\usepackage{mathptmx}
\usepackage{fourier}  % Use the Adobe Utopia font for the document - comment this line to return to the LaTeX default

% ---- Idioma --------

\usepackage[spanish, es-tabla]{babel} % Selecciona el español para palabras introducidas automáticamente, p.ej. "septiembre" en la fecha y especifica que se use la palabra Tabla en vez de Cuadro

% ---- Otros paquetes ----

\usepackage{url} % ,href} %para incluir URLs e hipervínculos dentro del texto (aunque hay que instalar href)
\usepackage{amsmath,amsfonts,amsthm} % Math packages
%\usepackage{graphics,graphicx, floatrow} %para incluir imágenes y notas en las imágenes
\usepackage{graphics,graphicx, float} %para incluir imágenes y colocarlas
\graphicspath{ {images/} }
\usepackage{subfig}

\usepackage{algorithm}
\usepackage{algpseudocode}

\usepackage{wrapfig}

% Para hacer tablas comlejas
\usepackage{multirow}
%\usepackage{threeparttable}

%\usepackage{sectsty} % Allows customizing section commands
%\allsectionsfont{\centering \normalfont\scshape} % Make all sections centered, the default font and small caps

\usepackage{fancyhdr} % Custom headers and footers
\pagestyle{fancyplain} % Makes all pages in the document conform to the custom headers and footers
\fancyhead{} % No page header - if you want one, create it in the same way as the footers below
\fancyfoot[L]{} % Empty left footer
\fancyfoot[C]{} % Empty center footer
\fancyfoot[R]{\thepage} % Page numbering for right footer
\renewcommand{\headrulewidth}{0pt} % Remove header underlines
\renewcommand{\footrulewidth}{0pt} % Remove footer underlines
\setlength{\headheight}{13.6pt} % Customize the height of the header

\numberwithin{equation}{section} % Number equations within sections (i.e. 1.1, 1.2, 2.1, 2.2 instead of 1, 2, 3, 4)
\numberwithin{figure}{section} % Number figures within sections (i.e. 1.1, 1.2, 2.1, 2.2 instead of 1, 2, 3, 4)
\numberwithin{table}{section} % Number tables within sections (i.e. 1.1, 1.2, 2.1, 2.2 instead of 1, 2, 3, 4)

\setlength\parindent{0pt} % Removes all indentation from paragraphs - comment this line for an assignment with lots of text

\newcommand{\horrule}[1]{\rule{\linewidth}{#1}} % Create horizontal rule command with 1 argument of height


\everymath{\displaystyle}
%----------------------------------------------------------------------------------------
%	TÍTULO Y DATOS DEL ALUMNO
%----------------------------------------------------------------------------------------

\title{	
\normalfont \normalsize 
\textsc{\textbf{Metaheurística} \\ Grado en Ingeniería Informática \\ Universidad de Granada} \\ [25pt] % Your university, school and/or department name(s)
\horrule{0.5pt} \\[0.4cm] % Thin top horizontal rule
\huge Práctica Alternativa ALO: Ant Lion Optimizer \\ % The assignment title
\horrule{2pt} \\[0.5cm] % Thick bottom horizontal rule
}

\author{José Luis Molina Aguilar} % Nombre y apellidos

\date{\normalsize\today} % Incluye la fecha actual


%----------------------------------------------------------------------------------------
% DOCUMENTO
%----------------------------------------------------------------------------------------

\begin{document}


\maketitle % Muestra el Título
  \begin{large}
    \centering
  \vfill
  
  Curso 2021-2022\\
  DNI : 77556436E\\
  Correo : joselu201@correo.ugr.es\\
  Grupo : A3, MARTES 17:30 - 19:30
  \vfill
  \end{large}
\newpage %inserta un salto de página

\tableofcontents % para generar el índice de contenidos

\listoffigures


\newpage



%----------------------------------------------------------------------------------------
%	Cuestión 1
%----------------------------------------------------------------------------------------

\section{ALO: Ant Lion Optimizer}
En este documento veremos y explicaremos una Metaheurística basada en ................ y la modelaremos
para aplicarlo al problema de practicas de MDD.



\subsection{Inspiracion}
  ALO esta inspitada en el modelo de caza que realiza la hormiga leon contra insectos, especificamente contra hormigas.
  Los hormigas leon tienen dos fases principales en su vida, la fase de larva y su fase adulta, es es su fase de larva en la que
  que nos fijamos para desarrolar ALO, las hormigas leon tiene ese nombre devido a que en su fase de larva teien una forma 
  pareceida a las de las hormigas.\\
  Su metodo de caza consiste en escaver un foso en el suelo arenoso con forma de cono, de forma que todo lo que caiga en 
  ese foso caera abajo, justo al lugar dondo la larva de hormiga leon se mantiene esperando enterrada.
  La hormiga leon es un depredador pasivo ya que espera en el fondo del foso esperando a que las presas ciagan dentro
  Aqune tambien tienen otras estrategias para la caza como poderosas mandibulas con veneno incluso son capaces de tirar arena hacia 
  la presa con el fin de hacerlas deslicer al centro.\\

  Esta Metaheurística simula este comportamiento siguiente los siguientes pasos:
  \begin{itemize}
    \item Desplazamiento aleatorio de las hormigas
    \item Atrapar a las presas en el trampa
    \item Construir la trampa
    \item Deslizamiento hacia la hormiga leon
    \item Capturando la presa y Recostruyendo el foso
    
  \end{itemize}


  \subsection{Desplazamiento aleatorio de las hormigas}

  Como podemos ver en la naturaleza, el desplazamiento de las hormigas por la superficie es aleatorio, esto nos ayuda 
  añadiendo una gran componente de diversidad a la hora de crear soluciones.

  En la implentacion original de ALO, el movimiento de las hormigas esta acotado para mantener las restricciones de las variables 
  del problema\\
  Para simular caer en una trampa se restringe el movimiento aleatoria de las hormigsa conforma avanzan las iteracioens, esto permite tener en un principio mucha diversidad y conforme vamos
  avanzando vamos especificando en la solucion.\\

  Entonces para calcular la neva posicion de la hormiga: \\
  \begin{equation}
    X_i^t = \frac{(X_i^t -a_i ) \cdot (d_i -c_i^t)}{(d_i -a_i)} + c_i
  \end{equation}

  donde $a$ es la cota inferior y $b$ es la cota superior de las variables mientras que $c$ y $d$ es el rango de movimiento
  $X_i^t $ es una suma acumulativa de la variable $i$ en la iteracion $y$
  
  \subsection{Atrapar a las presas en el trampa}

  Como hemos dicho antes el desplazamiento de las hormigas se ve influenciado por las trampas de la hormiga leon

  \begin{equation}
    c_i^t = Antlion_j^t +c^t
    d_i^t = Antlion_j^t +d^t
  \end{equation}

  donde $c^t$ es el minimo de todas las variables en esa iteracion, mientras que $d_i^t$ representa el maximo 
  de las varaibles en esa iteracion.\\
  De esta forma definimos un moviemeinto circular alrededor del foso de la hormiga leon seleccionada

  \subsection{Construir la trampa}

  En cada iteracion escogemos una hormiga leon para cada una de las hormigas, esta seleccion las realizamos mediente una 
  Ruleta la que toma en cuante el fitness de cada hormiga leon y el fitness acumulado de todas ellas. 
  De esta forma conseguirmos que las hormiga leon con mejor fitness tengas mas probabiliadeds de cazar hormigas

  \subsection{Deslizamiento hacia la hormiga leon}
   Para simular el deslizamiento que realizar la hormiga leon con diferentes metodos sobre las hormigas se simula reduciendo el radio
   de movimiento de las hormigas:\\

   \begin{equation}
    c^t = \frac{c^t}{I}
    d^t = \frac{d^t}{I}
  \end{equation}

  donde I es el radio y se calcula respeco de las iteraciones.

  \subsection{Capturando la presa u Recostruyendo el foso}
  La ultima parte de la caza es cuando la hormiga llega al fondo del foso y es atrapada por la hormiga leon.
  Asumimos la captura se lleva a cabo cuando el fitness de la hormige se haga mejor que el de su correspondiente hormiga leon.\\
  Despues actualizamos la posicion de la hormiga leon a la ultima posicion de la hormiga presa

  \begin{algorithm}[H]
    \begin{algorithmic}[1]
      \If{$f(Ant_i^t) > f(AntLion_j^t)$}
        \State $AntLion_j^t \gets Ant_i^t$
      \EndIf
    \end{algorithmic}
  \end{algorithm}

  \subsection{Elitismo}
  Vamos a aplicar elitismo para encontrar la mejor solucion. Para ello en cada iteracion la mejor hormiga leon se 
  guardara como elite y esta afectara los movimientos del resto de hormigas, por ello
  \begin{equation}
   Ant_i^t = \frac{R_A^t + R_E^t}{2}
  \end{equation}

  donde $R_A^t$ es el movimiento aleatoria alrededor de la hormiga leon seleccionada mediante la ruleta, mientras que
   $R_E^t$ es el moviemeinto aleatorio alrededor de la elite

\end{document}
