%%%%%%%%%%%%%%%%%%%%%%%%%%%%%%%%%%%%%%%%%
% Short Sectioned Assignment LaTeX Template Version 1.0 (5/5/12)
% This template has been downloaded from: http://www.LaTeXTemplates.com
% Original author:  Frits Wenneker (http://www.howtotex.com)
% License: CC BY-NC-SA 3.0 (http://creativecommons.org/licenses/by-nc-sa/3.0/)
%%%%%%%%%%%%%%%%%%%%%%%%%%%%%%%%%%%%%%%%%

%----------------------------------------------------------------------------------------
%	PACKAGES AND OTHER DOCUMENT CONFIGURATIONS
%----------------------------------------------------------------------------------------

\documentclass[paper=a4, fontsize=11pt]{scrartcl} % A4 paper and 11pt font size

% ---- Entrada y salida de texto -----

%\usepackage[T1]{fontenc} % Use 8-bit encoding that has 256 glyphs
\usepackage[utf8]{inputenc}

\usepackage{mathptmx}
\usepackage{fourier}  % Use the Adobe Utopia font for the document - comment this line to return to the LaTeX default

% ---- Idioma --------

\usepackage[spanish, es-tabla]{babel} % Selecciona el español para palabras introducidas automáticamente, p.ej. "septiembre" en la fecha y especifica que se use la palabra Tabla en vez de Cuadro

% ---- Otros paquetes ----

\usepackage{url} % ,href} %para incluir URLs e hipervínculos dentro del texto (aunque hay que instalar href)
\usepackage{amsmath,amsfonts,amsthm} % Math packages
%\usepackage{graphics,graphicx, floatrow} %para incluir imágenes y notas en las imágenes
\usepackage{graphics,graphicx, float} %para incluir imágenes y colocarlas
\graphicspath{ {images/} }
\usepackage{subfig}

\usepackage{algorithm}
\usepackage{algpseudocode}

\usepackage{wrapfig}

% Para hacer tablas comlejas
\usepackage{multirow}
%\usepackage{threeparttable}

%\usepackage{sectsty} % Allows customizing section commands
%\allsectionsfont{\centering \normalfont\scshape} % Make all sections centered, the default font and small caps

\usepackage{fancyhdr} % Custom headers and footers
\pagestyle{fancyplain} % Makes all pages in the document conform to the custom headers and footers
\fancyhead{} % No page header - if you want one, create it in the same way as the footers below
\fancyfoot[L]{} % Empty left footer
\fancyfoot[C]{} % Empty center footer
\fancyfoot[R]{\thepage} % Page numbering for right footer
\renewcommand{\headrulewidth}{0pt} % Remove header underlines
\renewcommand{\footrulewidth}{0pt} % Remove footer underlines
\setlength{\headheight}{13.6pt} % Customize the height of the header

\numberwithin{equation}{section} % Number equations within sections (i.e. 1.1, 1.2, 2.1, 2.2 instead of 1, 2, 3, 4)
\numberwithin{figure}{section} % Number figures within sections (i.e. 1.1, 1.2, 2.1, 2.2 instead of 1, 2, 3, 4)
\numberwithin{table}{section} % Number tables within sections (i.e. 1.1, 1.2, 2.1, 2.2 instead of 1, 2, 3, 4)

\setlength\parindent{0pt} % Removes all indentation from paragraphs - comment this line for an assignment with lots of text

\newcommand{\horrule}[1]{\rule{\linewidth}{#1}} % Create horizontal rule command with 1 argument of height


\everymath{\displaystyle}
%----------------------------------------------------------------------------------------
%	TÍTULO Y DATOS DEL ALUMNO
%----------------------------------------------------------------------------------------

\title{	
\normalfont \normalsize 
\textsc{\textbf{Técnicas de los Sistemas Inteligentes} \\ Grado en Ingeniería Informática \\ Universidad de Granada} \\ [25pt] % Your university, school and/or department name(s)
\horrule{0.5pt} \\[0.4cm] % Thin top horizontal rule
\huge Práctica 3: Representación de dominios y resolución de problemas con técnicas de planificación  \\ % The assignment title
\horrule{2pt} \\[0.5cm] % Thick bottom horizontal rule
}

\author{José Luis Molina Aguilar} % Nombre y apellidos

\date{\normalsize\today} % Incluye la fecha actual


%----------------------------------------------------------------------------------------
% DOCUMENTO
%----------------------------------------------------------------------------------------

\begin{document}


\maketitle % Muestra el Título
  \begin{large}
    \centering
  \vfill
  
  Curso 2021-2022\\
  Correo : joselu201@correo.ugr.es
  \vfill
  \end{large}
\newpage %inserta un salto de página

\tableofcontents % para generar el índice de contenidos

\listoftables


\newpage



%----------------------------------------------------------------------------------------
%	Cuestión 1
%----------------------------------------------------------------------------------------
\newpage
\section{Tabla de Resultados}
En esta tabla podemos ver el numero de acciones que se han tomado para conseguir su objetivo,
seguido del tiempo empleado por MetricFF, en el ejercicio 8 ademas se incluye el coste.
\begin{table}[H]
  \centering
  \begin{tabular}{|c|c|c|}
  \hline
  \textbf{Ejercicio} & \textbf{Nº Acciones}      & \textbf{Time} \\ \hline
  \textbf{1}         & \textbf{3}                & \textbf{0.00}    \\ \hline
  \textbf{2}         & \textbf{11}               & \textbf{0.00}    \\ \hline
  \textbf{3}         & \textbf{16}               & \textbf{0.01} \\ \hline
  \textbf{4}         & \textbf{28}               & \textbf{0.01} \\ \hline
  \textbf{5}         & \textbf{24}               & \textbf{0.00} \\ \hline
  \textbf{6}         & \textbf{24}               & \textbf{0.02} \\ \hline
  \textbf{7}         & \textbf{45}               & \textbf{0.7}  \\ \hline
  \textbf{8}         & \textbf{46 (Coste = 493)} & \textbf{0.2}  \\ \hline
  \end{tabular}
  \end{table}


Como podemos ver por la tabla anterior conforme aumentamos la complejidad del sistema
vemos que tarda más tiempo, esto se debe a : \\
"seconds searching, evaluating 21073 states, to a max depth of 0"\\
Conforme más estados busquemos mas tiempo tardara porque esa variable es la que más tiempo consume de media
en la resolución del ejercicio.
Esto se incrementa cuando intentamos buscar óptimos ya que tendra que buscar en un espacio de estados mucho mayor.
  
que le cuesta encontrar soluciones cada vez más, sobre todo si se exige la solución óptima.
  

\section{Pregunta 1:  }
En mi caso podemos ver que en la mayoria de llamadas obtengo el mismo tiempo 0.01 (+/- 0.01), esta poca 
variacion cambiando el coste se debe a que ya en principio encuentra el plan optimo sin necesidad de acudir a 
la modificacion del coste, sin embargo si no se consiguiese el optimo tendriamos que ampliar el espacio de busqueda 
lo cual aumentaria el tiempo de computo.

\section{Pregunta 2:}


\end{document}
