\input{preambuloSimple.tex}

\everymath{\displaystyle}
%----------------------------------------------------------------------------------------
%	TÍTULO Y DATOS DEL ALUMNO
%----------------------------------------------------------------------------------------

\title{	
\normalfont \normalsize 
\textsc{\textbf{Metaheurística} \\ Grado en Ingeniería Informática \\ Universidad de Granada} \\ [25pt] % Your university, school and/or department name(s)
\horrule{0.5pt} \\[0.4cm] % Thin top horizontal rule
\huge Práctica 1.a Técnicas de Búsqueda Local y Algoritmos Greedy para el Problema de la Mínima Dispersión Diferencial \\ % The assignment title
\horrule{2pt} \\[0.5cm] % Thick bottom horizontal rule
}

\author{José Luis Molina Aguilar} % Nombre y apellidos

\date{\normalsize\today} % Incluye la fecha actual


%----------------------------------------------------------------------------------------
% DOCUMENTO
%----------------------------------------------------------------------------------------

\begin{document}


\maketitle % Muestra el Título
  \centering
  \vfill
  \begin{large}
  Curso 2021-2022\\
  DNI : 77556436E\\
  Correo : joselu201@correo.ugr.es\\
  Grupo : A3, MARTES 17:30 - 19:30
  \vfill
  \end{large}
\newpage %inserta un salto de página

\tableofcontents % para generar el índice de contenidos

\listoffigures

\listoftables

\newpage



%----------------------------------------------------------------------------------------
%	Cuestión 1
%----------------------------------------------------------------------------------------

\section{Descripción Problema de Mínima Dispersión Diferencial}
El problema de Mínima Dispersión Diferencial es un problema de optimización combinatoria que entra en la clase de problemas
\textbf{NP-Completo}
Este es un problema en el que las heuristica obtinen buenas soluciones en menos tiempo.

\subsection{Descripción}
Dado un conjunto de $n$ elementos todos ellos conectados entre si, representado por una matriz de distancias, obtener
un subconjunto $m$ talque la diferencia entre la maxima suma y la minima suma de los elementos de $m$
se minimiza
El conjunto $m < n$\\
$ m \subset n \mid Minimize DD(S_m) $\\
donde $DD(S_m) $ es la Dispersión Diferencial del conjunto de Soluciones de tamaño $m$\\
$sum dist_{u,S-u}$  donde $dist_{u,S-u}$ son las distancias des un punto $u$ al resto de los puntos.\\

$DD(S) = max{sum dist_{u,S-u}} - min{sum dist_{u,S-u}}$
\section{Greedy}

\section{Busqueda por Trayectorias Simples (BL)}






%------------------------------------------------

\bibliography{citas} %archivo citas.bib que contiene las entradas 
\bibliographystyle{plain} % hay varias formas de citar

\end{document}
