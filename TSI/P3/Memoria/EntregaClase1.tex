\input{preambuloSimple.tex}

\everymath{\displaystyle}
%----------------------------------------------------------------------------------------
%	TÍTULO Y DATOS DEL ALUMNO
%----------------------------------------------------------------------------------------

\title{	
\normalfont \normalsize 
\textsc{\textbf{Técnicas de los Sistemas Inteligentes} \\ Grado en Ingeniería Informática \\ Universidad de Granada} \\ [25pt] % Your university, school and/or department name(s)
\horrule{0.5pt} \\[0.4cm] % Thin top horizontal rule
\huge Práctica 3: Representación de dominios y resolución de problemas con técnicas de planificación  \\ % The assignment title
\horrule{2pt} \\[0.5cm] % Thick bottom horizontal rule
}

\author{José Luis Molina Aguilar} % Nombre y apellidos

\date{\normalsize\today} % Incluye la fecha actual


%----------------------------------------------------------------------------------------
% DOCUMENTO
%----------------------------------------------------------------------------------------

\begin{document}


\maketitle % Muestra el Título
  \begin{large}
    \centering
  \vfill
  
  Curso 2021-2022\\
  Correo : joselu201@correo.ugr.es
  \vfill
  \end{large}
\newpage %inserta un salto de página

\tableofcontents % para generar el índice de contenidos

\listoftables


\newpage



%----------------------------------------------------------------------------------------
%	Cuestión 1
%----------------------------------------------------------------------------------------
\newpage
\section{Tabla de Resultados}
En esta tabla podemos ver el numero de acciones que se han tomado para conseguir su objetivo,
seguido del tiempo empleado por MetricFF, en el ejercicio 8 ademas se incluye el coste.
\begin{table}[H]
  \centering
  \begin{tabular}{|c|c|c|}
  \hline
  \textbf{Ejercicio} & \textbf{Nº Acciones}      & \textbf{Time} \\ \hline
  \textbf{1}         & \textbf{3}                & \textbf{0.00}    \\ \hline
  \textbf{2}         & \textbf{11}               & \textbf{0.00}    \\ \hline
  \textbf{3}         & \textbf{16}               & \textbf{0.01} \\ \hline
  \textbf{4}         & \textbf{28}               & \textbf{0.01} \\ \hline
  \textbf{5}         & \textbf{24}               & \textbf{0.00} \\ \hline
  \textbf{6}         & \textbf{24}               & \textbf{0.02} \\ \hline
  \textbf{7}         & \textbf{45}               & \textbf{0.7}  \\ \hline
  \textbf{8}         & \textbf{46 (Coste = 493)} & \textbf{0.2}  \\ \hline
  \end{tabular}
  \end{table}


Como podemos ver por la tabla anterior conforme aumentamos la complejidad del sistema
vemos que tarda más tiempo, esto se debe a : \\
"seconds searching, evaluating 21073 states, to a max depth of 0"\\
Conforme más estados busquemos mas tiempo tardara porque esa variable es la que más tiempo consume de media
en la resolución del ejercicio.
Esto se incrementa cuando intentamos buscar óptimos ya que tendra que buscar en un espacio de estados mucho mayor.
  
que le cuesta encontrar soluciones cada vez más, sobre todo si se exige la solución óptima.
  

\section{Pregunta 1:  }
En mi caso podemos ver que en la mayoria de llamadas obtengo el mismo tiempo 0.01 (+/- 0.01), esta poca 
variacion cambiando el coste se debe a que ya en principio encuentra el plan optimo sin necesidad de acudir a 
la modificacion del coste, sin embargo si no se consiguiese el optimo tendriamos que ampliar el espacio de busqueda 
lo cual aumentaria el tiempo de computo.

\section{Pregunta 2:}


\end{document}
